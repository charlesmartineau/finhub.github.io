% Options for packages loaded elsewhere
\PassOptionsToPackage{unicode}{hyperref}
\PassOptionsToPackage{hyphens}{url}
\PassOptionsToPackage{dvipsnames,svgnames,x11names}{xcolor}
%
\documentclass[
  letterpaper,
  DIV=11,
  numbers=noendperiod]{scrartcl}

\usepackage{amsmath,amssymb}
\usepackage{iftex}
\ifPDFTeX
  \usepackage[T1]{fontenc}
  \usepackage[utf8]{inputenc}
  \usepackage{textcomp} % provide euro and other symbols
\else % if luatex or xetex
  \usepackage{unicode-math}
  \defaultfontfeatures{Scale=MatchLowercase}
  \defaultfontfeatures[\rmfamily]{Ligatures=TeX,Scale=1}
\fi
\usepackage{lmodern}
\ifPDFTeX\else  
    % xetex/luatex font selection
\fi
% Use upquote if available, for straight quotes in verbatim environments
\IfFileExists{upquote.sty}{\usepackage{upquote}}{}
\IfFileExists{microtype.sty}{% use microtype if available
  \usepackage[]{microtype}
  \UseMicrotypeSet[protrusion]{basicmath} % disable protrusion for tt fonts
}{}
\makeatletter
\@ifundefined{KOMAClassName}{% if non-KOMA class
  \IfFileExists{parskip.sty}{%
    \usepackage{parskip}
  }{% else
    \setlength{\parindent}{0pt}
    \setlength{\parskip}{6pt plus 2pt minus 1pt}}
}{% if KOMA class
  \KOMAoptions{parskip=half}}
\makeatother
\usepackage{xcolor}
\setlength{\emergencystretch}{3em} % prevent overfull lines
\setcounter{secnumdepth}{3}
% Make \paragraph and \subparagraph free-standing
\ifx\paragraph\undefined\else
  \let\oldparagraph\paragraph
  \renewcommand{\paragraph}[1]{\oldparagraph{#1}\mbox{}}
\fi
\ifx\subparagraph\undefined\else
  \let\oldsubparagraph\subparagraph
  \renewcommand{\subparagraph}[1]{\oldsubparagraph{#1}\mbox{}}
\fi

\usepackage{color}
\usepackage{fancyvrb}
\newcommand{\VerbBar}{|}
\newcommand{\VERB}{\Verb[commandchars=\\\{\}]}
\DefineVerbatimEnvironment{Highlighting}{Verbatim}{commandchars=\\\{\}}
% Add ',fontsize=\small' for more characters per line
\usepackage{framed}
\definecolor{shadecolor}{RGB}{241,243,245}
\newenvironment{Shaded}{\begin{snugshade}}{\end{snugshade}}
\newcommand{\AlertTok}[1]{\textcolor[rgb]{0.68,0.00,0.00}{#1}}
\newcommand{\AnnotationTok}[1]{\textcolor[rgb]{0.37,0.37,0.37}{#1}}
\newcommand{\AttributeTok}[1]{\textcolor[rgb]{0.40,0.45,0.13}{#1}}
\newcommand{\BaseNTok}[1]{\textcolor[rgb]{0.68,0.00,0.00}{#1}}
\newcommand{\BuiltInTok}[1]{\textcolor[rgb]{0.00,0.23,0.31}{#1}}
\newcommand{\CharTok}[1]{\textcolor[rgb]{0.13,0.47,0.30}{#1}}
\newcommand{\CommentTok}[1]{\textcolor[rgb]{0.37,0.37,0.37}{#1}}
\newcommand{\CommentVarTok}[1]{\textcolor[rgb]{0.37,0.37,0.37}{\textit{#1}}}
\newcommand{\ConstantTok}[1]{\textcolor[rgb]{0.56,0.35,0.01}{#1}}
\newcommand{\ControlFlowTok}[1]{\textcolor[rgb]{0.00,0.23,0.31}{#1}}
\newcommand{\DataTypeTok}[1]{\textcolor[rgb]{0.68,0.00,0.00}{#1}}
\newcommand{\DecValTok}[1]{\textcolor[rgb]{0.68,0.00,0.00}{#1}}
\newcommand{\DocumentationTok}[1]{\textcolor[rgb]{0.37,0.37,0.37}{\textit{#1}}}
\newcommand{\ErrorTok}[1]{\textcolor[rgb]{0.68,0.00,0.00}{#1}}
\newcommand{\ExtensionTok}[1]{\textcolor[rgb]{0.00,0.23,0.31}{#1}}
\newcommand{\FloatTok}[1]{\textcolor[rgb]{0.68,0.00,0.00}{#1}}
\newcommand{\FunctionTok}[1]{\textcolor[rgb]{0.28,0.35,0.67}{#1}}
\newcommand{\ImportTok}[1]{\textcolor[rgb]{0.00,0.46,0.62}{#1}}
\newcommand{\InformationTok}[1]{\textcolor[rgb]{0.37,0.37,0.37}{#1}}
\newcommand{\KeywordTok}[1]{\textcolor[rgb]{0.00,0.23,0.31}{#1}}
\newcommand{\NormalTok}[1]{\textcolor[rgb]{0.00,0.23,0.31}{#1}}
\newcommand{\OperatorTok}[1]{\textcolor[rgb]{0.37,0.37,0.37}{#1}}
\newcommand{\OtherTok}[1]{\textcolor[rgb]{0.00,0.23,0.31}{#1}}
\newcommand{\PreprocessorTok}[1]{\textcolor[rgb]{0.68,0.00,0.00}{#1}}
\newcommand{\RegionMarkerTok}[1]{\textcolor[rgb]{0.00,0.23,0.31}{#1}}
\newcommand{\SpecialCharTok}[1]{\textcolor[rgb]{0.37,0.37,0.37}{#1}}
\newcommand{\SpecialStringTok}[1]{\textcolor[rgb]{0.13,0.47,0.30}{#1}}
\newcommand{\StringTok}[1]{\textcolor[rgb]{0.13,0.47,0.30}{#1}}
\newcommand{\VariableTok}[1]{\textcolor[rgb]{0.07,0.07,0.07}{#1}}
\newcommand{\VerbatimStringTok}[1]{\textcolor[rgb]{0.13,0.47,0.30}{#1}}
\newcommand{\WarningTok}[1]{\textcolor[rgb]{0.37,0.37,0.37}{\textit{#1}}}

\providecommand{\tightlist}{%
  \setlength{\itemsep}{0pt}\setlength{\parskip}{0pt}}\usepackage{longtable,booktabs,array}
\usepackage{calc} % for calculating minipage widths
% Correct order of tables after \paragraph or \subparagraph
\usepackage{etoolbox}
\makeatletter
\patchcmd\longtable{\par}{\if@noskipsec\mbox{}\fi\par}{}{}
\makeatother
% Allow footnotes in longtable head/foot
\IfFileExists{footnotehyper.sty}{\usepackage{footnotehyper}}{\usepackage{footnote}}
\makesavenoteenv{longtable}
\usepackage{graphicx}
\makeatletter
\def\maxwidth{\ifdim\Gin@nat@width>\linewidth\linewidth\else\Gin@nat@width\fi}
\def\maxheight{\ifdim\Gin@nat@height>\textheight\textheight\else\Gin@nat@height\fi}
\makeatother
% Scale images if necessary, so that they will not overflow the page
% margins by default, and it is still possible to overwrite the defaults
% using explicit options in \includegraphics[width, height, ...]{}
\setkeys{Gin}{width=\maxwidth,height=\maxheight,keepaspectratio}
% Set default figure placement to htbp
\makeatletter
\def\fps@figure{htbp}
\makeatother

\KOMAoption{captions}{tableheading}
\makeatletter
\makeatother
\makeatletter
\makeatother
\makeatletter
\@ifpackageloaded{caption}{}{\usepackage{caption}}
\AtBeginDocument{%
\ifdefined\contentsname
  \renewcommand*\contentsname{Table of contents}
\else
  \newcommand\contentsname{Table of contents}
\fi
\ifdefined\listfigurename
  \renewcommand*\listfigurename{List of Figures}
\else
  \newcommand\listfigurename{List of Figures}
\fi
\ifdefined\listtablename
  \renewcommand*\listtablename{List of Tables}
\else
  \newcommand\listtablename{List of Tables}
\fi
\ifdefined\figurename
  \renewcommand*\figurename{Figure}
\else
  \newcommand\figurename{Figure}
\fi
\ifdefined\tablename
  \renewcommand*\tablename{Table}
\else
  \newcommand\tablename{Table}
\fi
}
\@ifpackageloaded{float}{}{\usepackage{float}}
\floatstyle{ruled}
\@ifundefined{c@chapter}{\newfloat{codelisting}{h}{lop}}{\newfloat{codelisting}{h}{lop}[chapter]}
\floatname{codelisting}{Listing}
\newcommand*\listoflistings{\listof{codelisting}{List of Listings}}
\makeatother
\makeatletter
\@ifpackageloaded{caption}{}{\usepackage{caption}}
\@ifpackageloaded{subcaption}{}{\usepackage{subcaption}}
\makeatother
\makeatletter
\@ifpackageloaded{tcolorbox}{}{\usepackage[skins,breakable]{tcolorbox}}
\makeatother
\makeatletter
\@ifundefined{shadecolor}{\definecolor{shadecolor}{rgb}{.97, .97, .97}}
\makeatother
\makeatletter
\makeatother
\makeatletter
\makeatother
\ifLuaTeX
  \usepackage{selnolig}  % disable illegal ligatures
\fi
\IfFileExists{bookmark.sty}{\usepackage{bookmark}}{\usepackage{hyperref}}
\IfFileExists{xurl.sty}{\usepackage{xurl}}{} % add URL line breaks if available
\urlstyle{same} % disable monospaced font for URLs
\hypersetup{
  pdftitle={Setting up the Coding Environment},
  pdfauthor={Charles Martineau},
  colorlinks=true,
  linkcolor={blue},
  filecolor={Maroon},
  citecolor={Blue},
  urlcolor={Blue},
  pdfcreator={LaTeX via pandoc}}

\title{Setting up the Coding Environment}
\author{Charles Martineau}
\date{2024-09-09}

\begin{document}
\maketitle
\ifdefined\Shaded\renewenvironment{Shaded}{\begin{tcolorbox}[enhanced, sharp corners, interior hidden, borderline west={3pt}{0pt}{shadecolor}, frame hidden, breakable, boxrule=0pt]}{\end{tcolorbox}}\fi

\renewcommand*\contentsname{Table of contents}
{
\hypersetup{linkcolor=}
\setcounter{tocdepth}{2}
\tableofcontents
}
\hypertarget{poetry}{%
\section{Poetry}\label{poetry}}

We will work with poetry to manage the dependencies of the project.
Poetry is a tool for dependency management and packaging in Python. It
allows you to declare the libraries your project depends on and it will
manage (install/update) them for you.

To install Poetry with brew for \textbf{Mac} users, run the following
command in the Terminal app:

\texttt{brew\ install\ poetry}

For \textbf{Window} users, you can install Poetry by running the
following command in the command prompt:

\texttt{pip\ install\ poetry}

By default, Poetry installs Python for each project in the
\texttt{\textasciitilde{}/Library/Caches/pypoetry/virtualenvs/}
directory. I prefer to have it in the project directory. That way if, I
delete the directory, then the environment is deleted as well, which
prevents accumulating virtual environments for discarded projects. To
enable this, run the following command:

\texttt{poetry\ config\ virtualenvs.in-project\ true}

To create a poetry environment, first clone the repository and navigate
to the root directory of the project. Then run the following command:

\begin{Shaded}
\begin{Highlighting}[]
\NormalTok{cd }\OperatorTok{\textasciitilde{}/}\NormalTok{codes}\OperatorTok{/}  \CommentTok{\# where codes is the directory where you store the cloned repositories}
\NormalTok{poetry init}
\end{Highlighting}
\end{Shaded}

Once the project is created, you can add the dependencies:

\begin{Shaded}
\begin{Highlighting}[]
\NormalTok{poetry add pandas numpy scipy matplotlib seaborn statsmodels scikit}\OperatorTok{{-}}\NormalTok{learn linearmodels pyarrow jupyter pytest hydra}\OperatorTok{{-}}\NormalTok{core}
\end{Highlighting}
\end{Shaded}

You can always add more dependencies later by running the same command
with the additional dependencies.

This step updates the \texttt{pyproject.toml} file and creates a
\texttt{poetry.lock} file, which contains the exact version of each
dependency. This file is used to make sure that all collaborators use
the same version of each library. Note that because our dependencies are
built on top of other libraries, Poetry will also install the
dependencies of our dependencies.

To activate the environment in the terminal, run the following command:

\texttt{poetry\ shell}

Usually, after opening your GitHub repo through VS Code, the terminal
will show the name of the Python environment in the prompt that was
installed via Poetry. You won't need to run \texttt{poetry\ shell} in
this case.

If the \texttt{poetry.lock} file is already present when you clone the
repo, after install poetry, simply do

\texttt{poetry\ install}

\hypertarget{env-file}{%
\section{.env file}\label{env-file}}

We will use a \texttt{.env} file to store the environment variables.
This file will be used to store the API keys and other sensitive
information such as the root of the data directory. We will store the
data on a server (and maybe some on Dropbox) and the directory root to
the data is different for each user. The \texttt{.env} file should be in
the root directory of the project. The \texttt{.env} file should not be
committed to the repository. To prevent this, we will add \texttt{.env}
to the \texttt{.gitignore} file. The \texttt{.env} file should look like
this:

\begin{Shaded}
\begin{Highlighting}[]
\NormalTok{API\_KEY}\OperatorTok{=}\NormalTok{your\_api\_key}
\NormalTok{DATA\_DIR}\OperatorTok{=/}\NormalTok{path}\OperatorTok{/}\NormalTok{to}\OperatorTok{/}\NormalTok{data}
\NormalTok{FIG\_DIR}\OperatorTok{=}\NormalTok{.}\OperatorTok{/}\NormalTok{results}\OperatorTok{/}\NormalTok{figures}\OperatorTok{/}
\NormalTok{TBL\_DIR}\OperatorTok{=}\NormalTok{.}\OperatorTok{/}\NormalTok{results}\OperatorTok{/}\NormalTok{tables}\OperatorTok{/}
\NormalTok{TMP\_DIR}\OperatorTok{=}\NormalTok{.}\OperatorTok{/}\NormalTok{tmp}\OperatorTok{/}
\end{Highlighting}
\end{Shaded}

To read the environment variables, we will use the
\texttt{python-dotenv} library. To install it, run the following command
in terminal:

\texttt{poetry\ add\ python-dotenv}

To read the environment variables, the following code will be added to
the \texttt{main.py} file such that the environment variables are read
when the script is run:

\begin{Shaded}
\begin{Highlighting}[]
\ImportTok{from}\NormalTok{ dotenv }\ImportTok{import}\NormalTok{ load\_dotenv}
\ImportTok{from}\NormalTok{ pathlib }\ImportTok{import}\NormalTok{ Path }
\ImportTok{import}\NormalTok{ os}

\NormalTok{load\_dotenv()}

\NormalTok{api\_key }\OperatorTok{=}\NormalTok{ os.getenv(}\StringTok{"API\_KEY"}\NormalTok{)}

\NormalTok{datadir\_path }\OperatorTok{=}\NormalTok{ os.getenv(}\StringTok{"DATADIR"}\NormalTok{)}
\ControlFlowTok{if} \KeywordTok{not}\NormalTok{ datadir\_path:}
    \ControlFlowTok{raise} \PreprocessorTok{ValueError}\NormalTok{(}\StringTok{"DATADIR environment variable not set"}\NormalTok{)}

\NormalTok{data\_dir }\OperatorTok{=}\NormalTok{ Path(datadir\_path)}
\NormalTok{download\_dir }\OperatorTok{=}\NormalTok{ data\_dir }\OperatorTok{/} \StringTok{"download\_cache/"}
\NormalTok{open\_dir }\OperatorTok{=}\NormalTok{ data\_dir }\OperatorTok{/} \StringTok{"open/"}
\NormalTok{restricted\_dir }\OperatorTok{=}\NormalTok{ data\_dir }\OperatorTok{/} \StringTok{"restricted/"}
\NormalTok{clean\_dir }\OperatorTok{=}\NormalTok{ data\_dir }\OperatorTok{/} \StringTok{"clean/"}
\NormalTok{tmp\_dir }\OperatorTok{=}\NormalTok{ Path(os.getenv(}\StringTok{"TMP\_DIR"}\NormalTok{, }\StringTok{"tmp/"}\NormalTok{))}

\NormalTok{fig\_dir }\OperatorTok{=}\NormalTok{ Path(os.getenv(}\StringTok{"FIG\_DIR"}\NormalTok{, }\StringTok{"./results/figures/"}\NormalTok{))}
\NormalTok{tab\_dir }\OperatorTok{=}\NormalTok{ Path(os.getenv(}\StringTok{"TBL\_DIR"}\NormalTok{, }\StringTok{"./results/tables/"}\NormalTok{))}
\end{Highlighting}
\end{Shaded}

\hypertarget{hydra}{%
\section{Hydra}\label{hydra}}

We will use Hydra to manage the configuration of the project. Hydra is a
framework for elegantly configuring complex applications. It is used to
manage the configuration of the project. Hydra allows you to define a
configuration file with default values and then override these values
with command-line arguments. This is useful when you want to run the
same script with different parameters. To install Hydra, run the
following command in the terminal:

\texttt{poetry\ add\ hydra-core}

To use Hydra, you need to create a configuration file. The configuration
file should be in the \texttt{conf} directory. The configuration file
should be a YAML file. The configuration file should look like this:

\begin{Shaded}
\begin{Highlighting}[]
\FunctionTok{matplotlib}\KeywordTok{:}
\AttributeTok{  }\FunctionTok{font}\KeywordTok{:}
\AttributeTok{    }\FunctionTok{family}\KeywordTok{:}\AttributeTok{ serif}
\AttributeTok{    }\FunctionTok{sans\_serif}\KeywordTok{:}
\AttributeTok{      }\KeywordTok{{-}}\AttributeTok{ Helvetica}
\AttributeTok{    }\FunctionTok{serif}\KeywordTok{:}
\AttributeTok{      }\KeywordTok{{-}}\AttributeTok{ }\StringTok{"Computer Modern Roman"}

\FunctionTok{download\_data}\KeywordTok{:}
\AttributeTok{  }\FunctionTok{crsp}\KeywordTok{:}\AttributeTok{ }\CharTok{false}
\AttributeTok{  }\FunctionTok{compustat}\KeywordTok{:}\AttributeTok{ }\CharTok{false}

\FunctionTok{process\_raw\_data}\KeywordTok{:}
\AttributeTok{  }\FunctionTok{ibes\_sue}\KeywordTok{:}\AttributeTok{ }\CharTok{false}

\FunctionTok{database}\KeywordTok{:}
\CommentTok{  \# panel}
\AttributeTok{  }\FunctionTok{build\_panel\_db}\KeywordTok{:}\AttributeTok{ }\CharTok{false}
\AttributeTok{  }\FunctionTok{save\_panel\_db}\KeywordTok{:}\AttributeTok{ }\CharTok{false}
\AttributeTok{  }\FunctionTok{load\_panel\_db}\KeywordTok{:}\AttributeTok{ }\CharTok{true}

\FunctionTok{tasks}\KeywordTok{:}
\CommentTok{  \# {-}{-}{-}{-} Figures {-}{-}{-}{-}}
\AttributeTok{  }\FunctionTok{long\_short\_cumul\_ret\_fig}\KeywordTok{:}\AttributeTok{ }\CharTok{false}
\end{Highlighting}
\end{Shaded}

To read the configuration file, the following example code will be added
to the \texttt{main.py} file such that the configuration file is read
when the script is run:

\begin{Shaded}
\begin{Highlighting}[]
\ImportTok{import}\NormalTok{ hydra}
\ImportTok{from}\NormalTok{ omegaconf }\ImportTok{import}\NormalTok{ DictConfig, OmegaConf}

\AttributeTok{@hydra.main}\NormalTok{(version\_base}\OperatorTok{=}\VariableTok{None}\NormalTok{, config\_path}\OperatorTok{=}\StringTok{"./conf"}\NormalTok{, config\_name}\OperatorTok{=}\StringTok{"config"}\NormalTok{)}
\KeywordTok{def}\NormalTok{ my\_app(cfg: DictConfig):}

\NormalTok{    configure\_pyplot(}
\NormalTok{        font\_family}\OperatorTok{=}\NormalTok{cfg.matplotlib.font.family,}
\NormalTok{        font\_serif}\OperatorTok{=}\NormalTok{cfg.matplotlib.font.serif,}
\NormalTok{        font\_sans\_serif}\OperatorTok{=}\NormalTok{cfg.matplotlib.font.sans\_serif,}
\NormalTok{    )}

\NormalTok{    download\_files(}
\NormalTok{        crsp}\OperatorTok{=}\NormalTok{cfg.download\_data.crsp,}
\NormalTok{        compustat}\OperatorTok{=}\NormalTok{cfg.download\_data.compustat,}
\NormalTok{    )}

    \CommentTok{\# Process raw files}
\NormalTok{    process\_raw\_files(}
\NormalTok{        ibes\_sue}\OperatorTok{=}\NormalTok{cfg.process\_raw\_data.ibes\_sue,}
\NormalTok{    )}

    \ControlFlowTok{if}\NormalTok{ cfg.database.build\_panel\_db:}
\NormalTok{        logging.info(}\StringTok{"Build panel daily database"}\NormalTok{)}
\NormalTok{        panel\_db }\OperatorTok{=}\NormalTok{ create\_panel\_dataset(}
\NormalTok{            open\_dir}\OperatorTok{=}\NormalTok{open\_dir,}
\NormalTok{            restricted\_dir}\OperatorTok{=}\NormalTok{restricted\_dir,}
\NormalTok{            start\_date}\OperatorTok{=}\StringTok{"2012{-}01{-}01"}\NormalTok{,}
\NormalTok{            end\_date}\OperatorTok{=}\StringTok{"2022{-}12{-}31"}\NormalTok{,}
\NormalTok{        )}
    \ControlFlowTok{if}\NormalTok{ cfg.database.save\_panel\_db:}
\NormalTok{        logging.info(}\StringTok{"Saving panel daily database"}\NormalTok{)}
\NormalTok{        panel\_db.to\_parquet(clean\_dir }\OperatorTok{/} \StringTok{"panel\_db.parquet"}\NormalTok{)}
\end{Highlighting}
\end{Shaded}

\hypertarget{code-directory}{%
\section{Code directory}\label{code-directory}}

The code directory in Git should be structured as follows (with .py
examples):

\begin{Shaded}
\begin{Highlighting}[]
\NormalTok{main\_code}\OperatorTok{/}
\NormalTok{    conf}\OperatorTok{/}
\NormalTok{        config.yaml}
\NormalTok{    database}\OperatorTok{/}
        \FunctionTok{\_\_init\_\_}\NormalTok{.py}
\NormalTok{        download\_data.py}
\NormalTok{        process\_raw\_data.py}
\NormalTok{        build\_panel\_db.py}
\NormalTok{    figure\_codes}\OperatorTok{/}
        \FunctionTok{\_\_init\_\_}\NormalTok{.py}
\NormalTok{    table\_codes}\OperatorTok{/}
        \FunctionTok{\_\_init\_\_}\NormalTok{.py}
\NormalTok{        regression\_table\_1.py}
\NormalTok{        regression\_table\_2.py}
\NormalTok{    utils}\OperatorTok{/}
        \FunctionTok{\_\_init\_\_}\NormalTok{.py}
\NormalTok{        common\_functions.py}
\NormalTok{    tests}\OperatorTok{/}
        \FunctionTok{\_\_init\_\_}\NormalTok{.py}
\NormalTok{        test\_download\_data.py}
\NormalTok{        test\_process\_raw\_data.py}
\NormalTok{        test\_build\_panel\_db.py}
\end{Highlighting}
\end{Shaded}

\hypertarget{data-directory}{%
\section{Data directory}\label{data-directory}}

The data directory in our shared drive should be structured as follows:

\begin{Shaded}
\begin{Highlighting}[]
\NormalTok{data}\OperatorTok{/}
\NormalTok{    download\_cache}\OperatorTok{/}
    \BuiltInTok{open}\OperatorTok{/}
\NormalTok{    restricted}\OperatorTok{/}
\NormalTok{    clean}\OperatorTok{/}
\end{Highlighting}
\end{Shaded}

The \texttt{download\_cache} directory is used to store the raw data
files downloaded from the internet. The \texttt{open} directory is used
to store the raw data files that are open to the public. The
\texttt{restricted} directory is used to store the raw data files that
are restricted. The \texttt{clean} directory is used to store the
cleaned data files.

We will not save the raw and processed data on Git. We will use our
shared dropbox folder.

\hypertarget{saving-data}{%
\section{Saving data}\label{saving-data}}

We should save our data using parquet format.

\hypertarget{naming-convention}{%
\section{Naming convention}\label{naming-convention}}

For default variables, like model parameters we use capital letters.
E.g., N\_LOOPS = 40 for the default number of loops. For others, we use
small cap. E.g., n\_loops = 40 for the number of loops in a specific
case. We will store default variables in a separate file called
\texttt{parameters.py} in the \texttt{utils} directory.

\hypertarget{file-formatting-on-saving}{%
\section{File formatting on saving}\label{file-formatting-on-saving}}

We will use the black formatter to format our code. To install black in
VS Code add the extension \texttt{Black\ Formatter} to format the code
automatically. To enable this, go to the settings and search for
\texttt{format\ on\ save} and check the box.



\end{document}
